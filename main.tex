\PassOptionsToPackage{dvipsnames}{xcolor}
\documentclass[10pt,a4paper]{altacv}
\geometry{left=1cm,right=9cm,marginparwidth=6.8cm,marginparsep=1.2cm,top=1.25cm,bottom=1.25cm,footskip=2\baselineskip}

% Font ______________________________________________________________
\usepackage[T1]{fontenc}
\usepackage[utf8]{inputenc}
\usepackage[default]{lato}

% Colours ____________________________________________________________
\definecolor{Mulberry}{HTML}{72243D} % Subtitles II (Seasoned...), icons, Desenvolvedor de Software
\definecolor{SlateGrey}{HTML}{2E2E2E} % Subtitles I (Desenvolvedor Ruby...), name, phone, place
\definecolor{LightGrey}{HTML}{666666} % text
\colorlet{heading}{Sepia} % Titles (objetivos, experiencia, conhecimentos...)
\colorlet{accent}{Mulberry} % ????? Subtitles II (Seasoned...), icons, Desenvolvedor de Software
\colorlet{emphasis}{SlateGrey} % Subtitles I (Desenvolvedor Ruby...), name, phone, place
\colorlet{body}{LightGrey} % text

% Change the bullets for itemize and rating marker
% for \cvskill if you want to
\renewcommand{\itemmarker}{{\small\textbullet}}
\renewcommand{\ratingmarker}{\faCircle}
\addbibresource{sample.bib} % sample.bib contains your publications

\usepackage[colorlinks]{hyperref}

%%%%%%%%%%%%%%%%%%%%%%%%%%%%%%%%%%%%%%%%%%%%%%%%%%%%%%%%%%%%%%%%%%%%%%%%%%%%%%%%%%%%%%%%%%%%%%%%%%%%%%%%%

\begin{document}

% _________________________________________HEADER _______________________________________________________

% HEADER TITLE (NAME)
\name{Rodrigo Avellar de Muniagurria}

% HEADER SUBTITLE (JOB)
\tagline{Desenvolvedor de Software | Professor | Produtor musical}

% HEADER PICTURE
\photo{2.5cm}{foto}

% HEADER CONTACTS AND LINKS
\personalinfo{
  %\mailaddress{José do Patrocínio, 9999/999}
  \email{\href{mailto:ramprofissinoal@gmail.com}{ramprofissinoal@gmail.com}}\hspace{9.5mm}
  \phone{(51)98193-7660}\hspace{14mm}
  \location{Foz do Iguaçu, PR | Brasil}
  \linkedin{\href{https://linkedin.com/in/rodrigo-avellar}{linkedin.com/in/rodrigo-avellar}}\hspace{5mm}
  \github{\href{http://github.com/roavellarm}{github.com/roavellarm}}\hspace{5mm}
  \homepage{\href{https://www.rodrigoavellar.com}{rodrigoavellar.com}}
}

%% Make the header extend all the way to the right, if you want. 
\begin{fullwidth}

\makecvheader

\end{fullwidth}

%% Depending on your tastes, you may want to make fonts of itemize environments slightly smaller
% \AtBeginEnvironment{itemize}{\small}

% ________________________________________ OBJETIVOS ___________________________________________________

\cvsection[page1sidebar]{Objetivos}
\begin{itemize}

\item Atuar na área de desenvolvimento de software.
\item Trabalhar em equipes que seguem principios orientados por práticas \textit{agile}, \textit{DevOps/Agile}, SCRUM, LEAN, entre outros.
\item Participar de projetos que contribuam na solução de problemas e necessidades de pessoas, instituições, empresas e/ou o ecossistema.
\end{itemize}

\medskip


% ________________________________________ EXPERIENCIA _________________________________________________

\cvsection{Experiência}

\cvevent
{Desenvolvedor Ruby e Ruby on Rails - Backend}
{Seasoned}
{ 2018 -2019 }{}
%% {Rua dos Andradas, 1180 - Porto Alegre}
\begin{itemize}
\item Desenvolvimento de APIs para sistemas web e mobile.
\end{itemize}
\divider

% _________________________________________________________

\cvevent
{Docente - Curso Tecnólogo de Produção de Espetáculos}
{Faculdade Monteiro Lobato}
{2012 - 2020}{}
%% {Rua dos Andradas, 1180 - Porto Alegre}
\begin{itemize}
\item Elementos Estéticos e Técnicos da Música
\item Fundamentos da trilha sonora, acústica e sonorização
\end{itemize}
\divider

% _________________________________________________________

\cvevent
{Estágio - \textit{Cloud Computing}}
{CePES – Centro de Pesquisa em Engenharia de Software - PUCRS}
{ 2017 }{}
%% {Av. Ipiranga, 6681 - Prédio 32,sala 110}
Grupo de pesquisa em Redes, Infraestrutura e computação em Nuvem (GRIN) da PUCRS - Programa PDTI - Projeto Dell Cloud.
\divider

% _________________________________________________________

\cvevent
{Pesquisa CNPq (Mestrado)}
{Programa de Pós-graduação em Música - UFRGS}
{2008 -- 2010}{}
%% {Rua Professor Annes Dias, 112}
\begin{itemize}
\item Pesquisas na área da Música Eletroacústica Mista e estética da música.
\end{itemize}
\divider

% _________________________________________________________

\cvevent
{Integrante de Grupo de Pesquisa}
{Centro de Música Eletrônica (CME) - UFRGS}
{2003 -- 2006}{}
%% {Rua Senhor dos Passos, 248, 6º andar}
\begin{itemize}
\item Pesquisas em música computacional e eletroacústica.
\item Escrita de artigo da área de música e tecnologia.
\item Ministrante de cursos de extensão do CME.
\end{itemize}
\divider

% _________________________________________________________

\cvevent
{Produtor Musical | Engenheiro de som}
{(Autônomo)}
{2003 -- 2020}{}
\begin{itemize}
\item Produção musical em estúdios. Gravação, mixagem e masterização.
\item Produção de gravações de concertos e shows (DVDs ao vivo).
\end{itemize}

\cvsection [page2sidebar]{Contribuições às Comunidades}
\wheelchart{1.5cm}{0.5cm}{%
  40/10em/accent!50/Produção de músicos independentes de baixa renda, 
  45/9em/accent!80/Organização de concertos de música contemporânea para a cidade de Porto Alegre,
  15/8em/accent!20/Contribuição em projeto OpenSource
}

\medskip

% ________________________________________ PROJETOS ________________________________________________

\cvsection
{Projetos}

\cvevent
{Sistema de Espacialização Sonora para Clínica}
{OuveBem - Clínica e Aparelhos Auditivos}
{2017}{}
%% {Rua Padre Chagas, 80}
Descrição: Concepção, arquitetura, projeto e implantação de sistema octafônico de alto-falantes de espacialização sonora para clínica de fonoaudiologia.

\divider

\cvevent
{Projeto Recém Doutor - Pesquisa em Música Eletroacústica Experimental}
{Fundação de Amparo à Pesquisa do Estado do Rio Grande do Sul - FAPERGS}
{2004 -- 2005}{}
Descrição: Criar catálogo de sons para composição eletroacústica. Utilizar recursos computacionais para transformar sons através de técnicas de síntese, sampler e processamento de áudio. Compor com o auxílio do computador um estudo de música eletroacústica experimental. Graduando: Rodrigo Avellar de Muniagurria. Responsável: Eloi Fernando Fritsch.

\medskip

% _____________________________________________________________________________________________________
% \cvsection{A Day of My Life}

% Adapted from @Jake's answer from http://tex.stackexchange.com/a/82729/226
% \wheelchart{outer radius}{inner radius}{
% comma-separated list of value/text width/color/detail}
% \wheelchart{1.5cm}{0.5cm}{%
%   3/8em/accent!30/{Sleep,\\beautiful sleep}, 
%   3/7em/accent!40/Hopeful novelist by night,
%   8/8em/accent!60/Daytime job,
%   2/10em/accent/Sports and relaxation,
%   5/6em/accent!20/Spending time with family
% }

% ________________________________________ PUBLICAÇÕES ________________________________________________
\cvsection{Publicações}
\nocite{*}

\printbibliography[heading=pubtype,title={\printinfo{\faFileTextO}{Artigos}},type=article]
\divider

% _________________________________________________________

\printbibliography[heading=pubtype,title={\printinfo{\faGroup}{Participação em Congressos e Palestras}},type=inproceedings]

% _________________________________________________________

% \addnextpagesidebar[-1ex]{page3sidebar}
\end{document}
