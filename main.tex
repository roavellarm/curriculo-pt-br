%%%%%%%%%%%%%%%%%
% This is an sample CV template created using altacv.cls
% (v1.1.4, 27 July 2018) written by LianTze Lim (liantze@gmail.com). Now compiles with pdfLaTeX, XeLaTeX and LuaLaTeX.
% 
%% It may be distributed and/or modified under the
%% conditions of the LaTeX Project Public License, either version 1.3
%% of this license or (at your option) any later version.
%% The latest version of this license is in
%%    http://www.latex-project.org/lppl.txt
%% and version 1.3 or later is part of all distributions of LaTeX
%% version 2003/12/01 or later.
%%%%%%%%%%%%%%%%

%% If you need to pass whatever options to xcolor
\PassOptionsToPackage{dvipsnames}{xcolor}

%% If you are using \orcid or academicons
%% icons, make sure you have the academicons 
%% option here, and compile with XeLaTeX
%% or LuaLaTeX.
% \documentclass[10pt,a4paper,academicons]{altacv}

%% Use the "normalphoto" option if you want a normal photo instead of cropped to a circle
% \documentclass[10pt,a4paper,normalphoto]{altacv}

\documentclass[10pt,a4paper]{altacv}
%% AltaCV uses the fontawesome and academicon fonts
%% and packages. 
%% See texdoc.net/pkg/fontawecome and http://texdoc.net/pkg/academicons for full list of symbols.
%% 
%% Compile with LuaLaTeX for best results. If you
%% want to use XeLaTeX, you may need to install
%% Academicons.ttf in your operating system's font 
%% folder.


% Change the page layout if you need to
\geometry{left=1cm,right=9cm,marginparwidth=6.8cm,marginparsep=1.2cm,top=1.25cm,bottom=1.25cm,footskip=2\baselineskip}

% Change the font if you want to.

% If using pdflatex:
\usepackage[T1]{fontenc}
\usepackage[utf8]{inputenc}
\usepackage[default]{lato}

% If using xelatex or lualatex:
% \setmainfont{Lato}

% Change the colours if you want to
\definecolor{Mulberry}{HTML}{72243D}
\definecolor{SlateGrey}{HTML}{2E2E2E}
\definecolor{LightGrey}{HTML}{666666}
\colorlet{heading}{Sepia}
\colorlet{accent}{Mulberry}
\colorlet{emphasis}{SlateGrey}
\colorlet{body}{LightGrey}

% Change the bullets for itemize and rating marker
% for \cvskill if you want to
\renewcommand{\itemmarker}{{\small\textbullet}}
\renewcommand{\ratingmarker}{\faCircle}
%% sample.bib contains your publications
\addbibresource{sample.bib}

\usepackage[colorlinks]{hyperref}

\begin{document}

\name{Rodrigo Avellar de Muniagurria}
\tagline{Desenvolvedor de Software | Professor | Produtor musical}
\photo{2.5cm}{foto}
\personalinfo{%
  % Not all of these are required!
  % You can add your own with \printinfo{symbol}{detail}
  \email{\href{mailto:ramprofissinoal@gmail.com}{ramprofissinoal@gmail.com}}\hspace{5mm}
  \phone{(51)98193-7660}\hspace{21mm}
%  \mailaddress{José do Patrocínio, 1050/31}
  \location{Porto Alegre, RS | Brasil}
  \homepage{\href{https://www.rodrigoavellar.com}{rodrigoavellar.com}}\hspace{0}
  \linkedin{\href{https://linkedin.com/in/rodrigo-avellar}{linkedin.com/in/rodrigo-avellar}}\hspace{1mm}
  \github{\href{http://github.com/roavellarm}{github.com/roavellarm}}


  % \homepage{\href{http://lattes.cnpq.br/6921640383231668}{lattes/cnpq}}
  % \linkedin{\href{https://www.linkedin.com/in/guilherme-bertoni-machado/}{linkedin.com/in/guilherme-bertoni-machado}}
  % \faFacebook{\href{https://www.facebook.com/gbertonimachado}{ facebook.com/gbertonimachado}}
  % \github{\href{https://www.github.com/XXX}{github.com/XXXX}} 
  
  
  %% You MUST add the academicons option to \documentclass, then compile with LuaLaTeX or XeLaTeX, if you want to use \orcid or other academicons commands.
%   \orcid{orcid.org/0000-0000-0000-0000}
}

%% Make the header extend all the way to the right, if you want. 
\begin{fullwidth}

\makecvheader

\end{fullwidth}

%% Depending on your tastes, you may want to make fonts of itemize environments slightly smaller
% \AtBeginEnvironment{itemize}{\small}


%% Provide the file name containing the sidebar contents as an optional parameter to \cvsection.
%% You can always just use \marginpar{...} if you do
%% not need to align the top of the contents to any
%% \cvsection title in the "main" bar.
\cvsection[page1sidebar]{Objetivos}
\begin{itemize}

\item Atuar na área de desenvolvimento de software.
\item Trabalhar em equipes que seguem principios orientados por práticas \textit{agile}, \textit{DevOps/Agile}, SCRUM, LEAN, entre outros.
\item Participar de projetos que contribuam na solução de problemas e necessidades de pessoas, instituições, empresas e/ou o ecossistema.
\end{itemize}

\medskip

\cvsection{Experiência}

\cvevent
{Desenvolvedor Ruby e Ruby on Rails - Backend}
{Seasoned}
{Nov 2018}{}
%% {Rua dos Andradas, 1180 - Porto Alegre}
\begin{itemize}
\item Desenvolvimento de APIs para sistemas web e mobile.
\end{itemize}

\divider

\cvevent
{Docente - Curso Tecnólogo de Produção de Espetáculos}
{Faculdade Monteiro Lobato}
{Ago 2012}{}
%% {Rua dos Andradas, 1180 - Porto Alegre}
\begin{itemize}
\item Elementos Estéticos e Técnicos da Música
\item Fundamentos da trilha sonora, acústica e sonorização
\end{itemize}

\divider

\cvevent
{Estágio - \textit{Cloud Computing}}
{CePES – Centro de Pesquisa em Engenharia de Software - PUCRS}
{Mai 2017 -- Ago 2017}{}
%% {Av. Ipiranga, 6681 - Prédio 32,sala 110}
Grupo de pesquisa em Redes, Infraestrutura e computação em Nuvem (GRIN) da PUCRS - Programa PDTI - Projeto Dell Cloud.

\divider

\cvevent
{Pesquisa CNPq (Mestrado)}
{Programa de Pós-graduação em Música - UFRGS}
{2008 -- 2010}{}
%% {Rua Professor Annes Dias, 112}
\begin{itemize}
\item Pesquisas na área da Música Eletroacústica Mista e estética da música.
\end{itemize}

\divider

\cvevent
{Integrante de Grupo de Pesquisa}
{Centro de Música Eletrônica (CME) - UFRGS}
{2003 -- 2006}{}
%% {Rua Senhor dos Passos, 248, 6º andar}
\begin{itemize}
\item Pesquisas em música computacional e eletroacústica.
\item Escrita de artigo da área de música e tecnologia.
\item Ministrante de cursos de extensão do CME.
\end{itemize}

\divider

\cvevent
{Produtor Musical | Engenheiro de som}
{(Autônomo)}
{2003 -- 2018}{}
\begin{itemize}
\item Produção musical em estúdios. Gravação, mixagem e masterização.
\item Produção de gravações de concertos e shows (DVDs ao vivo).
\end{itemize}

\cvsection [page2sidebar]{Contribuições às Comunidades}
\wheelchart{1.5cm}{0.5cm}{%
  40/10em/accent!50/Produção de músicos independentes de baixa renda, 
  45/9em/accent!80/Organização de concertos de música contemporânea para a cidade de Porto Alegre,
  15/8em/accent!20/Contribuição em projeto OpenSource
}
\medskip

\cvsection
{Projetos}

\cvevent
{Sistema de Espacialização Sonora para Clínica}
{OuveBem - Clínica e Aparelhos Auditivos}
{2017}{}
%% {Rua Padre Chagas, 80}
Descrição: Concepção, arquitetura, projeto e implantação de sistema octafônico de alto-falantes de espacialização sonora para clínica de fonoaudiologia.

\divider

\cvevent
{Projeto Recém Doutor - Pesquisa em Música Eletroacústica Experimental}
{Fundação de Amparo à Pesquisa do Estado do Rio Grande do Sul - FAPERGS}
{2004 -- 2005}{}
Descrição: Criar catálogo de sons para composição eletroacústica. Utilizar recursos computacionais para transformar sons através de técnicas de síntese, sampler e processamento de áudio. Compor com o auxílio do computador um estudo de música eletroacústica experimental. Graduando: Rodrigo Avellar de Muniagurria. Responsável: Eloi Fernando Fritsch.

\medskip

% \cvsection{A Day of My Life}

% Adapted from @Jake's answer from http://tex.stackexchange.com/a/82729/226
% \wheelchart{outer radius}{inner radius}{
% comma-separated list of value/text width/color/detail}
% \wheelchart{1.5cm}{0.5cm}{%
%   3/8em/accent!30/{Sleep,\\beautiful sleep}, 
%   3/7em/accent!40/Hopeful novelist by night,
%   8/8em/accent!60/Daytime job,
%   2/10em/accent/Sports and relaxation,
%   5/6em/accent!20/Spending time with family
% }

% \clearpage
\cvsection{Publicações}
\nocite{*}
% \printbibliography[heading=pubtype,title={\printinfo{\faBook}{Books}},type=book]
% 
% \divider
% 
\printbibliography[heading=pubtype,title={\printinfo{\faFileTextO}{Artigos}},type=article]

\divider

\printbibliography[heading=pubtype,title={\printinfo{\faGroup}{Participação em Congressos e Palestras}},type=inproceedings]

%% If the NEXT page doesn't start with a \cvsection but you'd
%% still like to add a sidebar, then use this command on THIS
%% page to add it. The optional argument lets you pull up the 
%% sidebar a bit so that it looks aligned with the top of the
%% main column.
% \addnextpagesidebar[-1ex]{page3sidebar}
\end{document}
